\section{Introduction}
This report is an attempt to explain optimizations in presence of imperfectly nested loops in a crisp and clear way. It has been adapted from the sections $6.4$ and $6.5$ of the book \prettyref{bib:allenkennedy}.

\section{A Short Overview}
\begin{description}
    \item[Case 1] In the case of imperfectly nested loops (\prettyref{sec:imperfectloops}) \emph{where the outermost loop can be parallelized}, parallelize the loop and move further into its body to look for more optimizations.
    \item[Case 2] If the \emph{outermost loop cannot be parallelized}, then maximally distributing it around the statements in its body, can be an effective transformation. This step will create further loop nests, which may lead to the following cases:
        \begin{description}
        \item[Case 2.1] \emph{Perfectly nested loop nests.} In this case use the perfect nest loop optimization algorithm.
        \item[Case 2.1] \emph{Imperfectly nested loop nests.} This is a result of a \emph{tight recurrence} (a cyclic dependency) involving a statement and an inner loop. In this case it is best to leave the loop sequential and move into its body to look for other optimizations.
        \end{description}
    \item[Case 3] \emph{The outer loop can neither be parallelized nor distributed.} Then leave the loop sequential and move into its body to look for other possible optimizations.

\end{description}

\section{What is an Imperfect Loop Nest}\label{sec:imperfectloops}

\section{What is an Imperfect Loop Nest}
